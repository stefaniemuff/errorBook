\documentclass[]{book}
\usepackage{lmodern}
\usepackage{amssymb,amsmath}
\usepackage{ifxetex,ifluatex}
\usepackage{fixltx2e} % provides \textsubscript
\ifnum 0\ifxetex 1\fi\ifluatex 1\fi=0 % if pdftex
  \usepackage[T1]{fontenc}
  \usepackage[utf8]{inputenc}
\else % if luatex or xelatex
  \ifxetex
    \usepackage{mathspec}
  \else
    \usepackage{fontspec}
  \fi
  \defaultfontfeatures{Ligatures=TeX,Scale=MatchLowercase}
\fi
% use upquote if available, for straight quotes in verbatim environments
\IfFileExists{upquote.sty}{\usepackage{upquote}}{}
% use microtype if available
\IfFileExists{microtype.sty}{%
\usepackage{microtype}
\UseMicrotypeSet[protrusion]{basicmath} % disable protrusion for tt fonts
}{}
\usepackage[margin=1in]{geometry}
\usepackage{hyperref}
\hypersetup{unicode=true,
            pdftitle={Applied Error Modeling in Regression},
            pdfauthor={Stefanie Muff and Lukas F. Keller},
            pdfborder={0 0 0},
            breaklinks=true}
\urlstyle{same}  % don't use monospace font for urls
\usepackage{natbib}
\bibliographystyle{apalike}
\usepackage{color}
\usepackage{fancyvrb}
\newcommand{\VerbBar}{|}
\newcommand{\VERB}{\Verb[commandchars=\\\{\}]}
\DefineVerbatimEnvironment{Highlighting}{Verbatim}{commandchars=\\\{\}}
% Add ',fontsize=\small' for more characters per line
\usepackage{framed}
\definecolor{shadecolor}{RGB}{248,248,248}
\newenvironment{Shaded}{\begin{snugshade}}{\end{snugshade}}
\newcommand{\KeywordTok}[1]{\textcolor[rgb]{0.13,0.29,0.53}{\textbf{#1}}}
\newcommand{\DataTypeTok}[1]{\textcolor[rgb]{0.13,0.29,0.53}{#1}}
\newcommand{\DecValTok}[1]{\textcolor[rgb]{0.00,0.00,0.81}{#1}}
\newcommand{\BaseNTok}[1]{\textcolor[rgb]{0.00,0.00,0.81}{#1}}
\newcommand{\FloatTok}[1]{\textcolor[rgb]{0.00,0.00,0.81}{#1}}
\newcommand{\ConstantTok}[1]{\textcolor[rgb]{0.00,0.00,0.00}{#1}}
\newcommand{\CharTok}[1]{\textcolor[rgb]{0.31,0.60,0.02}{#1}}
\newcommand{\SpecialCharTok}[1]{\textcolor[rgb]{0.00,0.00,0.00}{#1}}
\newcommand{\StringTok}[1]{\textcolor[rgb]{0.31,0.60,0.02}{#1}}
\newcommand{\VerbatimStringTok}[1]{\textcolor[rgb]{0.31,0.60,0.02}{#1}}
\newcommand{\SpecialStringTok}[1]{\textcolor[rgb]{0.31,0.60,0.02}{#1}}
\newcommand{\ImportTok}[1]{#1}
\newcommand{\CommentTok}[1]{\textcolor[rgb]{0.56,0.35,0.01}{\textit{#1}}}
\newcommand{\DocumentationTok}[1]{\textcolor[rgb]{0.56,0.35,0.01}{\textbf{\textit{#1}}}}
\newcommand{\AnnotationTok}[1]{\textcolor[rgb]{0.56,0.35,0.01}{\textbf{\textit{#1}}}}
\newcommand{\CommentVarTok}[1]{\textcolor[rgb]{0.56,0.35,0.01}{\textbf{\textit{#1}}}}
\newcommand{\OtherTok}[1]{\textcolor[rgb]{0.56,0.35,0.01}{#1}}
\newcommand{\FunctionTok}[1]{\textcolor[rgb]{0.00,0.00,0.00}{#1}}
\newcommand{\VariableTok}[1]{\textcolor[rgb]{0.00,0.00,0.00}{#1}}
\newcommand{\ControlFlowTok}[1]{\textcolor[rgb]{0.13,0.29,0.53}{\textbf{#1}}}
\newcommand{\OperatorTok}[1]{\textcolor[rgb]{0.81,0.36,0.00}{\textbf{#1}}}
\newcommand{\BuiltInTok}[1]{#1}
\newcommand{\ExtensionTok}[1]{#1}
\newcommand{\PreprocessorTok}[1]{\textcolor[rgb]{0.56,0.35,0.01}{\textit{#1}}}
\newcommand{\AttributeTok}[1]{\textcolor[rgb]{0.77,0.63,0.00}{#1}}
\newcommand{\RegionMarkerTok}[1]{#1}
\newcommand{\InformationTok}[1]{\textcolor[rgb]{0.56,0.35,0.01}{\textbf{\textit{#1}}}}
\newcommand{\WarningTok}[1]{\textcolor[rgb]{0.56,0.35,0.01}{\textbf{\textit{#1}}}}
\newcommand{\AlertTok}[1]{\textcolor[rgb]{0.94,0.16,0.16}{#1}}
\newcommand{\ErrorTok}[1]{\textcolor[rgb]{0.64,0.00,0.00}{\textbf{#1}}}
\newcommand{\NormalTok}[1]{#1}
\usepackage{longtable,booktabs}
\usepackage{graphicx,grffile}
\makeatletter
\def\maxwidth{\ifdim\Gin@nat@width>\linewidth\linewidth\else\Gin@nat@width\fi}
\def\maxheight{\ifdim\Gin@nat@height>\textheight\textheight\else\Gin@nat@height\fi}
\makeatother
% Scale images if necessary, so that they will not overflow the page
% margins by default, and it is still possible to overwrite the defaults
% using explicit options in \includegraphics[width, height, ...]{}
\setkeys{Gin}{width=\maxwidth,height=\maxheight,keepaspectratio}
\IfFileExists{parskip.sty}{%
\usepackage{parskip}
}{% else
\setlength{\parindent}{0pt}
\setlength{\parskip}{6pt plus 2pt minus 1pt}
}
\setlength{\emergencystretch}{3em}  % prevent overfull lines
\providecommand{\tightlist}{%
  \setlength{\itemsep}{0pt}\setlength{\parskip}{0pt}}
\setcounter{secnumdepth}{5}
% Redefines (sub)paragraphs to behave more like sections
\ifx\paragraph\undefined\else
\let\oldparagraph\paragraph
\renewcommand{\paragraph}[1]{\oldparagraph{#1}\mbox{}}
\fi
\ifx\subparagraph\undefined\else
\let\oldsubparagraph\subparagraph
\renewcommand{\subparagraph}[1]{\oldsubparagraph{#1}\mbox{}}
\fi

%%% Use protect on footnotes to avoid problems with footnotes in titles
\let\rmarkdownfootnote\footnote%
\def\footnote{\protect\rmarkdownfootnote}

%%% Change title format to be more compact
\usepackage{titling}

% Create subtitle command for use in maketitle
\newcommand{\subtitle}[1]{
  \posttitle{
    \begin{center}\large#1\end{center}
    }
}

\setlength{\droptitle}{-2em}
  \title{Applied Error Modeling in Regression}
  \pretitle{\vspace{\droptitle}\centering\huge}
  \posttitle{\par}
\subtitle{An Introduction with Examples in R}
  \author{Stefanie Muff and Lukas F. Keller}
  \preauthor{\centering\large\emph}
  \postauthor{\par}
  \predate{\centering\large\emph}
  \postdate{\par}
  \date{2018-12-22}

\usepackage{booktabs}
\usepackage{amsthm}
\makeatletter
\def\thm@space@setup{%
  \thm@preskip=8pt plus 2pt minus 4pt
  \thm@postskip=\thm@preskip
}
\makeatother

\usepackage{amsthm}
\newtheorem{theorem}{Theorem}[chapter]
\newtheorem{lemma}{Lemma}[chapter]
\theoremstyle{definition}
\newtheorem{definition}{Definition}[chapter]
\newtheorem{corollary}{Corollary}[chapter]
\newtheorem{proposition}{Proposition}[chapter]
\theoremstyle{definition}
\newtheorem{example}{Example}[chapter]
\theoremstyle{definition}
\newtheorem{exercise}{Exercise}[chapter]
\theoremstyle{remark}
\newtheorem*{remark}{Remark}
\newtheorem*{solution}{Solution}
\begin{document}
\maketitle

{
\setcounter{tocdepth}{1}
\tableofcontents
}
\chapter*{Preface}\label{preface}
\addcontentsline{toc}{chapter}{Preface}

This is a \emph{first draft} of a book that deals with effects and cures
of measurement error in variables of regression models. The aim of the
book is not only to discuss a broad range of problems and biases that
are induced by measurement error, but mainly to bridge the gap between
theory and the applications. The idea is to provide a basic toolkit of
methods to make error modeling accessible to a broad audience in the
applied sciences. The many examples discussed and analyzed in the book
all come with the associated R code.

Interestingly, the presence and effects of measurement error and
misclassification in covariates and the response of regression models
have been recognized already more than a century ago (see e.g.
\ldots{}). Thanks to huge efforts of many researchers, the consequences
of ignoring measurement error or misclassification are known in many
settings, at least in theory. Moreover, a huge variety of methods to
appropriately deal with measurement error exist, and several textbooks
in statistics are devoted to the topic
\citep{fuller1987, gustafson2004, carroll.etal2006, yi2017}. Despite
this, most -- if not all -- error modelig methods go largely unused. Why
is this so? We can only hypothesize about the reasons, but the problem
seems to have many factes. On one hand, measurement error is often
nothing that seems worth paying attention to, and given that even most
introductory textbooks in applied statistics do not discuss measurement
error, it is not surprising that entire generations of young scientists
get educated in statistics and data analysis without ever having hard of
the problems it may cause. On the other hand, error modeling methods can
quickly become very challenging. Unless the problem is a very standard
case, it is often necessary to formulate a new model, and it may be all
but obvious what the model should be, let alone how to implement an
actual procedure to fit it. But even if the error model is relatively
simple, like a standard classical measurement model in a covariate of a
regression model (see Section \ref{sec:errortypes}), some extra-effort
and more specialist software packages are required. As a consequence,
the hurdle to get started with the proper handling of measurement error
in data is much higher than for standard regression analyses.

If you are reading these lines, we assume that you either have a very
specific measurement error problem at hand, or you would like to get a
gentle introduction into the topic and its applications. \ldots{}

When we say ``error'', we do not only mean actual mistakes in the data
that are used to fit regression models.

Kind of uncertainty, noise or imprecision that are present in the data
that we use to fit our models.

N. Breslow, \emph{Lessons in Biostatistics} (2014) \citep{breslow2014}
wrote

\begin{quote}
Obviously, {[}. . .{]} the \emph{best} method of dealing with
measurement error was to avoid it.
\end{quote}

I say:

\begin{quote}
The \emph{second best} method of dealing with measurement error is to
properly account for it.
\end{quote}

We might develop a package. In this case, the
\textbf{package-to-be-developed} package can be installed from CRAN or
Github:

\begin{Shaded}
\begin{Highlighting}[]
\KeywordTok{install.packages}\NormalTok{(}\StringTok{"package-to-be-developed"}\NormalTok{)}
\CommentTok{# or the development version}
\CommentTok{# devtools::install_github("stefaniemuff/package-to-be-developed")}
\end{Highlighting}
\end{Shaded}

Follow us on Twitter!
\href{https://twitter.com/stefaniemuff}{@StefanieMuff}
\href{https://twitter.com/lukasfkeller}{@LukasFKeller}

\chapter{Introduction}\label{intro}

\section{What is Error?}\label{what-is-error}

Explain that measurement error often comes in the shape of uncertainty,
which is present in almost all data.

\section{Why and When do I Have to
Worry?}\label{why-and-when-do-i-have-to-worry}

\begin{itemize}
\tightlist
\item
  Triple whammy of ME
\item
  When is error a problem?
\item
  Bias versus variance
\item
  Is it sometimes better not to model the error?
\end{itemize}

It is surprising how many phenomena in statistics and its applications
can be viewed through the measurement error lens. A prominant example is
the concept of heritability in genetics and evolutionary biology, as we
will explain in Section \ref{sec:heritability}.

\section{Organization and Take-Home Messages of This
Book}\label{organization-and-take-home-messages-of-this-book}

\section{Outlook}\label{outlook}

What we are going to do, outlook to chapters.

\chapter{Types of Errors}\label{types-of-errors}

Before we can start to speak about the effects of measurement error
(Chapter \protect\hyperlink{effects}{3}), or how to account for it
(Chapter \protect\hyperlink{accounting}{4}), we have to spend some time
to understand whan \emph{kind} of error we are talking about.

Dichotomy into classical vs Berkson error, continous vs categorial
variables, differential vs non-differential error. Maybe more?

\section{Continuous Variables}\label{sec:errortypes}

Two fundamentally different error types

\subsection{Classical Measurement
Error}\label{classical-measurement-error}

\subsection{Berkson Measurement Error}\label{berkson-measurement-error}

\section{Categorical and Count
Variables}\label{categorical-and-count-variables}

\section{Differential vs Non-Differential
Error}\label{differential-vs-non-differential-error}

\hypertarget{effects}{\chapter{The Effects of Measurement
Error}\label{effects}}

We will look into effects of ME in the linear regression case.

\section{Classical Measurement
Error}\label{classical-measurement-error-1}

\section{The Concept of Heritability, Regression to the Mean and
Measurement Error}\label{sec:heritability}

Geneticists, evolutionary biologists and animal breeders will be
familiar with the concept of \emph{heritability} \eqref{eq:heritability}.

\begin{equation}
h^2 = \frac{\sigma_A^2}{\sigma_A^2 + \sigma_E^2}
\label{eq:heritability}
\end{equation}

\begin{itemize}
\tightlist
\item
  Will use data in Figures \ref{fig:galton1} and \ref{fig:galton2} to
  explain regression to the mean
\end{itemize}

\begin{figure}

{\centering \includegraphics{error-intro-book_files/figure-latex/galton1-1} 

}

\caption{Data drawn from `http://www.math.uah.edu/stat/data/Galton.txt`}\label{fig:galton1}
\end{figure}

\begin{figure}

{\centering \includegraphics{error-intro-book_files/figure-latex/galton2-1} 

}

\caption{Data drawn from `http://www.math.uah.edu/stat/data/Galton.txt`}\label{fig:galton2}
\end{figure}

\citep{fuller1987, galton1886}

\section{Berkson Measurement Error}\label{berkson-measurement-error-1}

\section{Error in Categorical and Count
Variables}\label{error-in-categorical-and-count-variables}

\section{Error in the response}\label{error-in-the-response}

\hypertarget{accounting}{\chapter{Methods to Account for Measurement
Error}\label{accounting}}

\section{Bayesian Methods}\label{bayesian-methods}

\section{Simulation Extrapolation
(SIMEX)}\label{simulation-extrapolation-simex}

\chapter{Linear Regression Models}\label{LinReg}

\chapter{Generalized Linear (Mixed) Models}\label{GLMMs}

\section{Classical error}\label{classical-error}

\subsection{Error in a covariate}\label{error-in-a-covariate}

\begin{itemize}
\tightlist
\item
  Correlated covariates
\end{itemize}

\subsection{Error in the response}\label{error-in-the-response-1}

\section{Berkson error}\label{berkson-error}

\subsection{Error in a covariate}\label{error-in-a-covariate-1}

\subsection{Error in the response}\label{error-in-the-response-2}

\chapter{Survival Models}\label{Survival}

\chapter{Advanced Topics}\label{advancedTopics}

\section{Misalignement Error in Spatial
Data}\label{misalignement-error-in-spatial-data}

\bibliography{book.bib,packages.bib,literature.bib}


\end{document}
